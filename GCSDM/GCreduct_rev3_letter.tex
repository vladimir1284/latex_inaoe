\documentclass{letter}

\usepackage{hyperref}
\hypersetup{urlcolor=blue,colorlinks=true}	    % colores en vez de cajas en los enlaces
\usepackage[utf8]{inputenc}
\usepackage{xr}
\externaldocument{GCreduct_rev2}

\address{Instituto Nacional de Astrofísica,\\ Óptica y Electrónica \\ Luis Enrrique Erro 1 \\ Sta. Ma. Tonantzintla,\\ Puebla}
\begin{document}

\begin{letter}{}
  \opening{Dear Editor in Chief and Reviewers:}

  We are pleased to resubmit a new version of our paper INS-D-16-1281R1: A New Algorithm for Reduct Computation based on Gap Elimination and Attribute Contribution. We want to thank the constructive criticisms of the editor and reviewers as well as their comments and questions. We have modified the original paper accordingly to the reviewer comments and we hope that, in its present form, the paper can be accepted. We have addressed the editor and reviewers concerns as outlined below.

  \textbf{Editor:} 
  This version of the paper is significantly improved but, as you can see below, there are still some remarks from Reviewer \#3 and partially also Reviewer \#2. Please try to prepare one more revision.
  \begin{enumerate}
	\item Besides the comments from reviewers, please improve grammar (for example: ``there is not a list" should be replaced with ``there is no list"). \\
	\textbf{Authors’ response:} \\
	We have proofread the whole paper in order to improve the grammar in the current version of the paper.
		 
	\item I am also interested how you would compare your GCreduct algorithm with some very-well-known algorithms based on attribute orderings/ permutations? Please take a look at the following publication for further references with this respect:
  
	\small{\url{ http://www.sciencedirect.com/science/article/pii/S0888613X17301408}}\\
	\normalsize
	\textbf{Authors’ response:} \\
	We found that Stawicki et al. (2017) and also Slezak \& Janusz (2011) used modified versions of very-well-known algorithms based on attribute orderings/ permutations for computing bireducts. These works make reference to the algorithms reported by Bazan et al. (2000) which are implemented in ROSETTA. In the first paragraph of the ``Related Work" section, we comment about the low performance of these early algorithms (see page~\pageref{early}). For this reason, we did not included these algorithms in our experiments.
  \end{enumerate}
  
  \textbf{Reviewer \#1:}
  
  The authors have addressed all my concerns and I thus recommend for publication.\\
  \textbf{Authors’ response:} \\
  We thank the reviewer for this comment, we really appreciate his suggestions that helped us to improve our paper.
  
  \textbf{Reviewer \#2:}
  
  I think that some of my negative comments from the previous review still refer to the current version of the paper. However, it should be noted that the authors have greatly improved the quality of paper, by taking into account the comments of the reviewers. Therefore, after a long reflection, I came to the conclusion that the paper can be published in the INS.\\
  \textbf{Authors’ response:} \\
  We thank the reviewer for this comment, and for his time to look into the main ideas of our manuscript, as well as suggesting us important details to improve our paper.
  
  \textbf{Reviewer \#3:}
  
  The article ``A New Algorithm for Reduct Computation based on Gap Elimination and Attribute Contribution" submitted to the Information Sciences journal is not ready for publication and I recommend major revision of this work. In general, I think the article is valuable. However, I suggest a major review because, in my opinion, article requires a number of amendments in terms of its form and content that should be verified/re-reviewed after adjustments.
 
 \begin{enumerate}  
	\item The main contribution of this work is GCreduct - a new algorithm for computing all reducts of an information system, based on the pruning properties of gap elimination and attribute contribution that performs faster than all other recent reported alternatives in a specific kind of information systems. You motivated the study and, in Section~2, introduced readers to basic concepts of rough sets theory. In sec~3 you provided interesting review of related works connected to the problem of all reducts computation. I have one minor remark with regard to that part of the article: You use the term `information system' instead of `decision system' what may confuse readers.\\
	\textbf{Authors’ response:} \\
	We agree with this reviewer's remark. In order to provide clarity we replaced the term `information system' by `decision system' where appropriate.
	
	\item Next, you presented a set of theories and definition that are the foundation of GCreduct algorithm. In my opinion this Section requires a number of amendments:
	First of all, most of Propositions, Definitions and Proofs is unclear. In my opinion, the most confusing is the numbering of attributes in sorted list of attributes L and its relation to ``global" numbering in sorted \textit{SBDM}.  For example, from Definition~3 we may learn that attribute $j_{q}$ and $j_{q+1}$ may not be consecutive in \textit{SBDM}. However, in Proposition~1 I'm not so sure about the relation of $c_{p+1}$ and $c_{p}$ in \textit{SBDM}.\\
	\textbf{Authors’ response:}\\
	To attend this concern, we added the following sentence in the paragraph after Definition~3:\\
	\textit{Notice from the notation used above that $c_{j_q}$ and $c_{j_{q+1}}$ are consecutive in $L$ while $c_{j_q}$ and $c_{j_q+1}$ are consecutive in the \textit{SBDM}.}
	
	Let's take for instance $L = [c_1,c_4]$ where $c_{j_q} = c_1$. We have $c_{j_{q+1}} = c_4$ and $c_{j_q+1}=c_2$.
	
	\item Second, the structure of  Definitions, Propositions etc. should be improved. In current form reading them is very hard. Understanding them (the same as authors do) - barely possible.\\
	\textbf{Authors’ response:}\\
	We have modified the structure of Definitions, Propositions an Corollaries in order to improve their uniformity and clarity. We hope that in the current version of the paper they will be easier to read and understand. 
	
	\item Third, on page 17 you started to describe sorting procedure. However, the description explains sorting of only one row.\\
	\textbf{Authors’ response:}\\
	 We have modified this part of the paper explaining that in the same way as other algorithms previously reported in the literature, GCreduct arranges the \textit{SBDM} by moving to the first position one of the rows with the fewest number of 1's. Then, all the columns in which the first row has a 1 are moved to the left. We included the illustration of this process through the example shown in Table~\ref{tab:SSBDM1} (see page~\pageref{arrange}).
	
	\item In section~4.2 the GCreduct algorithm is presented. In my opinion, at least the main algorithm should be presented in a more abstract way. In its current form, it is very hard to follow the algorithm and its description.\\
	\textbf{Authors’ response:}\\
	We added an abstract description of the proposed algorithm in the second paragraph of the subsection~4.2 (see page~\pageref{abstarct}). We hope that in the current version, the full description of the algorithm can be better understood.

	\item In Sec. 5 you presented experimental results of algorithm performance against some of related algorithms. In my opinion, you presented review and experimental  study of performance of implementations (not algorithms). Presented experiments and results showed that implementation of GCreduct is faster in case of SBDM with density of 1s below 0.36.\\
	\textbf{Authors’ response:}\\
	\textit{Since these algorithms have exponential complexity, we perform a comparison through their implementations. Thus, all the conclusions drawn from these experiments refers rigorously to the implementations of the algorithms that we used.}
	
	To attend this concern, we added this comment to the first paragraph of the section ``Evaluation and Discussion":\\
	
	
	\item Lastly, the form of the article and the style of the language should be improved.\\
	\textbf{Authors’ response:}\\
	We have proofread the whole paper in order to improve the grammar and the style of the current version of the paper.
	
	
  \end{enumerate}     
  
  Best regards,

  The authors.
  
\end{letter}

%% References without bibTeX database:
\textbf{References}
\begin{enumerate}
	\item Bazan, J. G., Nguyen, H. S., Nguyen, S. H., Synak, P., \& Wróblewski, J. (2000). Rough Set Algorithms in Classification Problem. In L. Polkowski, S. Tsumoto, \& T. Y. Lin (Eds.), Rough Set Methods and Applications: New Developments in Knowledge Discovery in Information Systems (pp. 49--88).
	
	\item Slezak, D., \& Janusz, A. (2011). Ensembles of Bireducts: Towards Robust Classification and Simple Representation. Lecture Notes in Computer Science (Including Subseries Lecture Notes in Artificial Intelligence and Lecture Notes in Bioinformatics), 7105(May), 64--77.
	
	\item Stawicki, S., Slezak, D., Janusz, A., \& Widz, S. (2017). Decision bireducts and decision reducts -- a comparison. International Journal of Approximate Reasoning, 84, 75--109.
\end{enumerate}
	
	

	
\end{document}