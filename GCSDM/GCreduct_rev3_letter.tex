\documentclass{letter}

\usepackage{hyperref}
\hypersetup{urlcolor=blue,colorlinks=true}	    % colores en vez de cajas en los enlaces
\usepackage[utf8]{inputenc}
\usepackage{xr}
\externaldocument{GCreduct_rev3}
\newtheorem{definition}{Definition}

\address{Instituto Nacional de Astrofísica,\\ Óptica y Electrónica \\ Luis Enrrique Erro 1 \\ Sta. Ma. Tonantzintla,\\ Puebla}
\begin{document}

\begin{letter}{}
  \opening{Dear Editor in Chief and Reviewers:}

  We are pleased to resubmit a new version of our paper INS-D-16-1281R2: A New Algorithm for Reduct Computation based on Gap Elimination and Attribute Contribution. We want to thank the constructive criticisms of the editor and reviewers, as well as their comments and questions. We have modified the original paper according to these comments and we hope that, in its present form, the paper can be accepted. We have addressed the editor and reviewers concerns as outlined below.

  \textbf{Editor:} 
  The linguistic quality needs improvement. It is essential to make sure that  the manuscript reads smoothly - this definitely helps the reader fully appreciate your research findings. Consult a professional. 
  \begin{enumerate}
	\item Show all changes made to the revised manuscript. \\
	\textbf{Authors’ response:} \\
	The whole paper has been proofread in order to improve the grammar and style. All changes have been highlighted in yellow in the new version of our paper. The changes accomplished to this end are summarized below:
	\begin{itemize}		
		\item We modified the last two sentences of the abstract in order to improve its readability, as follows:
		
		\textcolor{blue}{Therefore, in this paper, we propose a new algorithm for computing all reducts of a decision system, based on the pruning properties of \textit{gap  elimination} and \textit{attribute contribution}, that uses simpler operations for candidate evaluation in order to reduce the runtime. Finally, the proposed algorithm is evaluated and compared with other state of the art algorithms, over synthetic and real decision systems.}
				 
		\item We have rewritten the last sentence of the subsection ``Positive Region" as follows:
		
	    \textcolor{blue}{The \textit{B-positive region of D}, denoted as $POS_B(D)$, is defined as the set of all objects in $U$ such that if two of them have the same value for every attribute in B, they belong to the same decision class.}
		
		\item We have improved the last paragraph in page 8, as:
		
		\textcolor{blue}{Following the traversing order of LEX, the CT\_EXT algorithm searches for testors without verifying the typical condition. In this way, a larger number of candidates is evaluated, in comparison to LEX; but the cost of each evaluation is lower.}
		
		\item The second sentence of the fourth paragraph of the subsection 4.1 ``Pruning Properties for GCreduct", was modified as:
		
		\textcolor{blue}{Notice that the order of the elements in the list is relevant.}
		 
		\item Another grammar correction was made in the fifth paragraph of the subsection 4.1 ``Pruning Properties for GCreduct", as it can be seen below:
		
		\textcolor{blue}{Let us also define an order relation $\prec$ over the set of ordered lists of attributes as follows.}
		
		\item Some modifications were introduced in Propositions~\ref{prop:gap}, \ref{prop:contrib}, \ref{prop:cumul}, \ref{prop:exclude} and \ref{prop:firstRow}, Corollary~\ref{coro:gap} and Definition~\ref{coro:gap}, to improve their grammar.  
		
		\item The penultimate paragraph of the subsection 4.1 ``Pruning Properties for GCreduct" was improved as follows:
		
		\textcolor{blue}{Based on this proposition, if we reach an ordered list $L$ satisfying $c_{j_0}=0$, the search can be stopped.}
		
		\item The last sentence of the first paragraph of the subsection 4.2 ``The GCreduct algorithm" was improved as follows:
		
	    \textcolor{blue}{Unlike other works [29, 13] where candidate evaluation is performed through operations with a high cost, GCreduct uses a simpler candidate evaluation based on gap elimination and attribute contribution to reduce the runtime.}
		
		\item We have improved the first sentence in the penultimate paragraph of page~18, as follows:
		
		\textcolor{blue}{The pseudocode for the exclusion evaluation is shown in Algorithm~\ref{alg:exclusion}.}
		
		\item We have also changed the last sentence of the first paragraph of page 20, as follows:
		
		\textcolor{blue}{Thus, there is no polynomial complexity algorithm for computing all the reducts of a decision system.}
		
		\item We have modified the last sentence of the proof of Proposition~\ref{prop:findall}, as follows:
		
		\textcolor{blue}{Since the attribute subsets discarded by the searching process are certainly not reducts, we can state that GCreduct finds all the reducts of a decision system.}
		
		\item The first sentence in the last paragraph of page~22 was modified as follows:
		
		\textcolor{blue}{The exclusion verification starts with $cm_B = em_B = (00000)$.}
		
		\item The third paragraph of the section 5 ``Evaluation and Discussion" was modified as follows:
		
		\textcolor{blue}{The source code of the four algorithms, as well as all the decision systems used in these experiments, are available at \url{http://ccc.inaoep.mx/~ariel/GCreduct}.}
		
		\item The second paragraph of the subsection 5.3 ``Evaluation Over Synthetic \textit{SBDMs}" was improved as follows:
		
		\textcolor{blue}{For clarity purposes, the 500 synthetic \textit{SBDMs} were divided into 15 bins by discretizing the range of densities, each bin having approximately 33 \textit{SBDMs}. Figure~\ref{fig:scattDensity} shows the average runtime for all  matrices in each bin for the three algorithms, as a function of the density of 1's in the synthetic \textit{SBDMs}. In this figure, the vertical bars show the standard deviation in each bin.}
		
		\item We have improved the third sentence of the first paragraph in page 28, as follows:
		
		\textcolor{blue}{This operation is executed more often in fast--BR than in GCreduct.}
		
		\item The last sentence of the second paragraph in page~29 was rewritten as follows:
		
		\textcolor{blue}{Moreover, from these conclusions a great advantage can be taken of selecting the appropriate algorithm for a specific decision system, since the density of 1's can be computed a priori with a relatively low computational cost.}
		
		\item The last sentence of the last paragraph of the subsection 5.3 ``Evaluation Over Synthetic \textit{SBDMs}" was modified as follows:
		
		\textcolor{blue}{However, in our experiments, for density values not too close to 0.36 these variations in the distribution of 1's within the \textit{SBDMs} had no effect on the determination of the fastest algorithm, as it can be also seen in Figure~\ref{fig:scattDensity}.}
		
		\item The second and third paragraph of the conclusions were modified to improve their grammar and clarity.
		
	\end{itemize}
		 
	\item Shorten the paper - 35 pages max.\\
	\textbf{Authors’ response:} \\
	The paper was modified to fit 35 pages. The following modifications were made to this end:
	\begin{itemize}
		\item We removed the third paragraph from the Introduction where we discussed some recent applications of the algorithms for computing all reducts.
		
		\item From the  Basic Concepts, we removed the second paragraph where we introduced the meaning of indistinguishable objects. This term will be not used further in the current version of the paper.

		
		\item We have shortened the ``Related Work" section by eliminating the discussion of some previous works that we found less related to our proposed algorithm.
		
		%\item From the subsection ``The GCreduct algorithm", we removed the proposition where we explicitly show that the algorithm GCreduct finds all the reducts of a decision system. 
		
		\item We have shorten the subsection ``GCreduct vs. fast--CT\_EXT and fast--BR" by including the information shown in Table~9 (Candidates evaluated by Fast--CT\_EXT, GCreduct and Fast--BR for decision systems from UCI) within Table~8 (Candidates evaluated by Fast--CT\_EXT, GCreduct and Fast--BR for decision systems from UCI). We also eliminated Figure~1 (GCreduct fraction of evaluated candidates vs. the number of attributes of the decision systems) since the information shown in this figure implicitly appears in the new Table~\ref{tab:java} (Runtime and candidates evaluated by Fast--CT\_EXT, GCreduct and Fast--BR for decision systems taken from UCI).
		
		
	\end{itemize}
  \end{enumerate}
  
  \textbf{Reviewer \#3:}
  
  The article ``A New Algorithm for Reduct Computation based on Gap Elimination and Attribute Contribution" submitted by you to the Information Sciences journal is, in my opinion, ready for publication and I recommended to accept this work. 
  
  In my opinion, you have greatly improved the quality of the paper, by taking into account the comments of the reviewers. Nevertheless, I was hesitating a bit, between the `minor revision' and `acceptance' of the paper since there are still a few, minor shortcomings. However, the list of my remarks is short and the severity of them is low, hence I believe that all shortcomings may be improved by you during the work on the camera ready version (if the final decision would be the acceptance of the article). Below you may find my remarks:  
 
 \begin{enumerate}  
	\item (page 5) you still confuse Information System with Decision System - see Table~1: ``Example of an Information System, where $c_0 - c_6$ are condition attributes and $d$ is a decision attribute".\\
	\textbf{Authors’ response:} \\	
	We agree this remark and we modified the caption of Table~\ref{tab_IS} as follows:
	
	\textcolor{blue}{``Example of a Decision System, where $c_0-c_6$ are conditional attributes and $d$ is a decision attribute.".}
	
	\item condition attributes $=>$ conditional attributes.\\
	\textbf{Authors’ response:}\\
	To attend this suggestion, we replaced condition attribute/s by conditional attribute/s wherever appropriate.
	
	\item (page 6) Definition 1 should be corrected: $r_1 => r_k.$\\
	\textbf{Authors’ response:}\\
	We specially thank the reviewer for this remark since this error in the definition would cause confusion to the reader. We made the corrections, and the definition in its new form is shown below.
	
	\begin{definition}\label{def:basic_row}
		\textcolor{blue}{Let $DM$ be a binary discernibility matrix and $r_k \in DM$ be a row of $DM$. We say that $r_k$ is a superfluous row of $DM$ if $\exists r \in DM$ such that $\exists i | (r[i] < r_k[i]) \wedge \forall i | (r[i] \leq r_k[i])$, where $r[i]$ is the $i$-th element of the row $r$.}	
	\end{definition}	 
	
	\item (page 35) There are still some issues with the article language, i.e.: ``within the SBDM have not effect on the determination of the fastest algorithm" $=>$ ..have no effect on the determination.. or ..have not affected the determination..\\
	\textbf{Authors’ response:}\\
	This sentence was corrected as follows:
	
	\textcolor{blue}{However, in our experiments, for density values not too close to 0.36 these variations in the distribution of 1's within the \textit{SBDMs} have no effect on the determination of the fastest algorithm, as it can be also seen in Figure~\ref{fig:scattDensity}.}
	
	\item (page 35) In my opinion the conclusion ``we can conclude that GCreduct performs faster than the fastest previously reported algorithms" should be rephrased to $=>$  ``..GCreduct performs faster than: fast-BR, ..."\\
	\textbf{Authors’ response:}\\
	We modified this sentence as suggested:
	
	\textcolor{blue}{After conducting a series of experiments over synthetic \textit{SBDMs} and decision systems from the UCI repository, we can conclude that GCreduct performs faster than fast--BR and fast--CT\_EXT on those decision systems whose associated \textit{SBDM} has a density of 1's under 0.36.}
	
  \end{enumerate}    
   
  We thank reviewer \#3 for his/her comments, and for taking the time for carefully reading our paper.
  
  Best regards,

  The authors.
  
\end{letter}

\end{document}