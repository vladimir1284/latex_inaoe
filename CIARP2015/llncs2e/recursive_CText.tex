% This is LLNCS.DEM the demonstration file of
% the LaTeX macro package from Springer-Verlag
% for Lecture Notes in Computer Science,
% version 2.4 for LaTeX2e as of 16. April 2010
%
\documentclass[citeauthoryear]{llncs}
%
%\usepackage{makeidx}  % allows for indexgeneration
%
\begin{document}
\mainmatter              % start of the contributions
%
\title{A Recursive Implementation of the CT\_EXT Algorithm 
	   for the Typical Testor Property Identification}
%
\titlerunning{Recursive CT_EXT}  % abbreviated title (for running head)
%                                     also used for the TOC unless
%                                     \toctitle is used
			 
\author{Vlad\'{i}mir Rodr\'{i}guez\inst{1,2} \and Jos\'{e}~F. Mart\'{i}nez\inst{1}
		 \and Jes\'{u}s~A. Carrasco\inst{1} \and Manuel~S.~Lazo\inst{1}}
%
\authorrunning{Vlad\'{i}mir Rodr\'{i}guez et al.} % abbreviated author list (for running head)
%
%%%% list of authors for the TOC (use if author list has to be modified)
%\tocauthor{Ivar Ekeland, Roger Temam, Jeffrey Dean, David Grove,
%Craig Chambers, Kim B. Bruce, and Elisa Bertino}
%
\institute{Instituto Nacional de Astrof\'{i}sica, \'{O}ptica y Electr\'{o}nica,\\
		   Luis Enrique Erro \# 1, Tonantzintla, Puebla, M\'{e}xico,\\
		   Coordinaci\'{o}n de Ciencias Computacionales,\\
\email{vladimir.rodriguez@ccc.inaoep.mx}
\and Universidad de Camag\"{u}ey,\\
	 Circunvalaci\'{o}n Nte. km 5$\frac{1}{2}$, Camag\"{u}ey, Cuba}


\maketitle              % typeset the title of the contribution

\begin{abstract}
	The Testors Theory is an important approach to feature selection 
	in supervised classification. Typical testors are irreducible subsets 
	of features preserving the discernibility between objects in the 
	original dataset. A new dataset using only those features in a 
	typical testor is a reduced representation of the original one,
	which improves the efficiency of machine learning tools without 
	degrading their performance. Finding the complete set of typical 
	testors for 	a dataset requires a high computational effort. Here, 
	we propose a recursive implementation of the CT\_EXT algorithm 
	(one of the fastest reported in the literature) which avoids some 
	unnecessary verifications, reducing that way its runtime execution. 
	An experimental comparison between our proposal and other state of 
	the art algorithms, over publicly available datasets, is presented.
\keywords{Testors Theory, features selection, CT\_EXT algorithm}
\end{abstract}
%
\section{Introduction}
%
	Feature selection is an important task in supervised classification. An 
	information system is a dataset (table) containing objects (rows) which are
	characterized by the value of some features (columns). A reduced 
	representation of an information system, with the same discernibility
	between objects than that of the original dataset, reduces the computational
	cost of the classification process. In the Logical Combinatorial Pattern 
	Recognition (Ruiz-Shulcloper et al.~\cite{Shulcloper1995}), Testors Theory 
	emerges as a solution to feature selection (Ruiz-Shulcloper~\cite{Shulcloper2008};
	Mart\'inez-Trinidad and	Guzm\'an-Arenas~\cite{Martinez2001}). A testor is subset of 
	features which allows us to discern between objects in different classes by 
	using only values of its features. A typical testor is defined as a testor which 
	is minimal with respect to inclusion. The main limitation to the application of
	the Testors Theory is that finding all the reducts from a dataset has been proven
	as an NP-hard problem (Skowron et al.~\cite{Skowron1992}).
	
	One of the first algorithm designed to overcome the exponential complexity (regarding
	the number of features) of the problem of finding all the typical testors, was 
	proposed by Ruiz-Shulcloper et al.~(\cite{Shulcloper1985}). This algorithm, called BT
	codified a subset of features as a binary word with as many bits as features in the 
	dataset. With a 0 representing the absence of the corresponding feature in the current
	subset, and a 1 representing its inclusion. This way, candidates subsets running from
	(0,...,0,1) to (1,...,1,0), in the natural order of binary numbers, are evaluated for 
	the testor condition. Notice that the empty subset (0,...,0) and the complete set of
	features (1,...,1) (obviously a testor) are not included. The pruning process in the
	search space is based on the minimal condition of typical testors and a convenient sorting
	of the basic matrix (see definition~\cite{def:BM}) associated to the dataset. Finally, 
	testors found by BT algorithm must by compared with each other in order to remove
	any superset (not a typical testor by definition).
	
	In (Ruiz-Shulcloper et al.~\cite{Shulcloper1995b}) a new algorithm (REC) is presented.
	The main drawback of REC was that it operated directly over the dataset (instead of the
	basic matrix), handling a huge amount of superfluous information. Ayaquica~\cite{Ayaquica1997}
	presented the algorithm CER directed to solve this problem by using a different traversing
	order. 
	
	Then, Santiesteban and Pons-Porrata~(\cite{Santiesteban2003}) proposed a revolutionary algorithm
	called LEX. Main ideas behind LEX are a new traversing order of candidates (which resembles the
	lexicographical order in which string characters are compared) and the concept of gap. In LEX
	the typical condition is verified first and only for those potentially typical testors, the testor 
	condition is checked. This way, the out-coming testors from this algorithm are always typical.
	The concept of gap allows us; once obtained a typical testor (or a not testor) candidate, including 
	the last feature in the dataset, avoid the evaluation of any subset of this candidate.
	
	Sanchez-D\'iaz and Lazo-Cort\'es~(Sanchez2007) proposed the CT\_EXT algorithm for computing all
	typical testors.
	
%
\section{Basic Concepts}
%
\subsection{Example}
%
\section{The Recursive Implementation of the CT\_EXT Algorithm}
%
\subsection{Example}
%
\section{Experiments}
%
%
\section{Conclusions}
%
% ---- Bibliography ----
%
\begin{thebibliography}{}
%


\bibitem[1997]{Ayaquica1997}
	Ayaquica, I. O.:
	Un nuevo algoritmo de escala exterior para el c\'alculo de testores t\'ipicos.
	Memorias del II Taller Iberoamericano de Reconocimiento de Patrones , La
	Habana. 141--148 (1997)

\bibitem[2007]{Sanchez2007}
	Sanchez-D\'iaz, G., Lazo-Cort\'es, M.:
	CT-EXT: an algorithm for computing typical testor set. 
	In Progress in Pattern Recognition, Image Analysis and Applications. Springer. 506--514 (2007)

\bibitem[2001]{Martinez2001}
	Mart\'inez-Trinidad, J.F., Guzm\'an-Arenas, A.: 
	The Logical Combinatorial Approach to Pattern Recognition an Overview through Selected Works. 
	Pattern Recognition. 34, 741--751 (2001)

\bibitem[2003]{Santiesteban2003}	
	Santiesteban, Y., Pons-Porrata, A.:
	LEX: a new algorithm for the calculus of typical testors. 
	Mathematics Sciences Journal, 21(1), 85--95 (2003)

\bibitem[1995a]{Shulcloper1995}
	Ruiz-Shulcloper, J., Alba-Cabrera, E., Lazo-Cort\'es, M.:
	Introducci\'{o}n al Reconocimiento de Patrones (Enfoque L\'{o}gico--Combinatorio). 
	Serie Verde No. 51. CINVESTAV-IPN, México, (1995)

\bibitem[1995b]{Shulcloper1995b}
	Ruiz-Shulcloper, J., Alba-Cabrera, E., Lazo-Cort\'es, M.:
	Introducci\'{o}n a la teor\'ia de Testotes T\'ipicos. 
	Serie Verde No. 50. CINVESTAV-IPN, México, (1995)
	
\bibitem[1985]{Shulcloper1985}	
	Ruiz-Shulcloper, J., Aguila, L., Bravo, A.:
	BT and TB algorithms for computing all irreducible testors. 
	Revista Ciencias Matem\'{a}ticas, 2, 11--18 (1985)

\bibitem[2008]{Shulcloper2008}
	Ruiz-Shulcloper, J.:
	Pattern recognition with mixed and incomplete data. 
	Pattern Recognition and Image Analysis. 18(4), 563--576 (2008)

\bibitem[1992]{Skowron1992}
	Skowron, A., Rauszer, C.:
	The discernibility matrices and functions in information systems. 
	Handbook of Applications and Advances of the Rough Sets Theory. 331--362  (1992)
	
	
\end{thebibliography}

%\clearpage
%\addtocmark[2]{Author Index} % additional numbered TOC entry
%\renewcommand{\indexname}{Author Index}
%\printindex
%\clearpage
%\addtocmark[2]{Subject Index} % additional numbered TOC entry
%\markboth{Subject Index}{Subject Index}
%\renewcommand{\indexname}{Subject Index}
%\input{subjidx.ind}
\end{document}
