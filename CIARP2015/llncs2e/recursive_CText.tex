% This is LLNCS.DEM the demonstration file of
% the LaTeX macro package from Springer-Verlag
% for Lecture Notes in Computer Science,
% version 2.4 for LaTeX2e as of 16. April 2010
%
\documentclass[citeauthoryear]{llncs}
%
%\usepackage{makeidx}  % allows for indexgeneration
%
\usepackage{algorithm,algorithmic}
\usepackage{multicol}

\begin{document}
\mainmatter              % start of the contributions
%
\title{A Recursive Implementation of the CT\_EXT Algorithm 
	   for the Testor Property Identification}
%
\titlerunning{Recursive CT_EXT}  % abbreviated title (for running head)
%                                     also used for the TOC unless
%                                     \toctitle is used
			 
\author{Vlad\'{i}mir Rodr\'{i}guez\inst{1,2} \and Jos\'{e}~F. Mart\'{i}nez\inst{1}
		 \and Jes\'{u}s~A. Carrasco\inst{1} \and Manuel~S.~Lazo\inst{1}}
%
\authorrunning{Vlad\'{i}mir Rodr\'{i}guez et al.} % abbreviated author list (for running head)
%
%%%% list of authors for the TOC (use if author list has to be modified)
%\tocauthor{Ivar Ekeland, Roger Temam, Jeffrey Dean, David Grove,
%Craig Chambers, Kim B. Bruce, and Elisa Bertino}
%
\institute{Instituto Nacional de Astrof\'{i}sica, \'{O}ptica y Electr\'{o}nica,\\
		   Luis Enrique Erro \# 1, Tonantzintla, Puebla, M\'{e}xico,\\
		   Coordinaci\'{o}n de Ciencias Computacionales,\\
\email{vladimir.rodriguez@ccc.inaoep.mx}
\and Universidad de Camag\"{u}ey,\\
	 Circunvalaci\'{o}n Nte. km 5$\frac{1}{2}$, Camag\"{u}ey, Cuba}


\maketitle              % typeset the title of the contribution

\begin{abstract}
	The Testors Theory is an important approach to feature selection 
	in supervised classification. Typical testors are irreducible subsets 
	of features preserving the discernibility between objects in the 
	original dataset. A new dataset using only those features in a 
	typical testor is a reduced representation of the original one,
	which improves the efficiency of machine learning tools without 
	degrading their performance. Finding the complete set of typical 
	testors for 	a dataset requires a high computational effort. Here, 
	we propose a recursive implementation of the CT\_EXT algorithm 
	(one of the fastest reported in the literature) which avoids some 
	unnecessary verifications, reducing that way its runtime execution. 
	An experimental comparison between our proposal and other state of 
	the art algorithms, over publicly available datasets, is presented.
\keywords{Testors Theory, features selection, CT\_EXT algorithm}
\end{abstract}
%
\section{Introduction}
%
	Feature selection is an important task in supervised classification. An 
	information system is a dataset (table) containing objects (rows) which are
	characterized by the value of some features (columns). A reduced 
	representation of an information system, with the same discernibility
	between objects than that of the original dataset, reduces the computational
	cost of the classification process. In the Logical Combinatorial Pattern 
	Recognition (Ruiz-Shulcloper et al.~\cite{Shulcloper1995}), Testors Theory 
	emerges as a solution to feature selection (Ruiz-Shulcloper~\cite{Shulcloper2008};
	Mart\'inez-Trinidad and	Guzm\'an-Arenas~\cite{Martinez2001}). A testor is subset of 
	features which allows us to discern between objects in different classes by 
	using only values of its features. A Typical Testor (TT) is defined as a testor which 
	is minimal with respect to inclusion. The main limitation to the application of
	the Testors Theory is that finding all the TT from a dataset has been proven
	as an NP-hard problem (Skowron et al.~\cite{Skowron1992}).
	
	One of the first algorithm designed to overcome the exponential complexity (regarding
	the number of features) of the problem of finding all the TT, was 
	proposed by Ruiz-Shulcloper et al.~(\cite{Shulcloper1985}). This algorithm, called BT,
	codified a subset of features as a binary word with as many bits as features in the 
	dataset. A 0 represents the absence of the corresponding feature in the current
	subset while a 1 represents its inclusion. This way, candidates subsets are evaluated
	in the natural order of binary numbers. The pruning process in the
	search space is based on the minimal condition of TT and a convenient sorting
	of the basic matrix (see definition~\ref{def:BM}) associated to the dataset. Finally, 
	testors found by BT algorithm must by compared with each other in order to remove
	any superset (not a TT by definition).
	In (Ruiz-Shulcloper et al.~\cite{Shulcloper1995b}) a new algorithm (REC) is presented.
	The main drawback of REC was that it operated directly over the dataset (instead of the
	basic matrix), handling a huge amount of superfluous information. Ayaquica~(\cite{Ayaquica1997})
	presented the algorithm CER directed to solve this problem by using a different traversing
	order. 
	
	Then, Santiesteban and Pons-Porrata~(\cite{Santiesteban2003}) proposed a revolutionary algorithm
	called LEX. Main ideas behind LEX are a new traversing order of candidates (which resembles the
	lexicographical order in which string characters are compared) and the concept of gap. In LEX
	the typical condition is verified first and only for those potentially TT, the testor 
	condition is checked. This way, the out-coming testors from this algorithm are always typical.
	The concept of gap allows us; once obtained a TT (or a not testor) candidate, including 
	the last feature in the dataset, avoid the evaluation of any subset of this candidate.
	
	Sanchez-D\'iaz and Lazo-Cort\'es~(\cite{Sanchez2007}) proposed the CT\_EXT algorithm for computing all
	TT. Following a traversing order similar to that in LEX, this algorithm search for
	testors without verifying the typical condition. This way, a larger number of candidates are 
	evaluated, in comparison to LEX; but the cost of each evaluation is lower. Results from experiments
	show that CT\_EXT is faster than the previous existing algorithm for most datasets. Then, Lias-Rodr\'iguez
	and Pons-Porrata~(\cite{Lias2009}) presented the BR algorithm, a Recursive algorithm based on 
	Binary operations. BR is very similar to LEX in its bones but its recursive nature encloses a great
	gain. Given a candidate subset, the remaining features are tested a priori and those being rejected are
	excluded from subsequent evaluations. Sanchez-D\'iaz et al.~(\cite{Sanchez2010}) presented a cumulative
	procedure for the CT\_EXT algorithm. This fast-CT\_EXT implementation reduces drastically the runtime
	for most datasets at no extra cost. In (Lias-Rodr\'iguez and Sanchez-D\'iaz~\cite{Lias2013}) the
	gap elimination and column reduction are added to BR. This fast-BR algorithm is, no doubt the one 
	evaluating the minimum number of candidates in the state of the art. The main drawback of fast-BR and 
	BR is, as in LEX, the high cost of evaluating the typical condition for every candidate. 
	
	We present in this paper recursive implementation of the fast-CT\_EXT algorithm. We show that, as 
	in BR, the a priori verification of features reduces the number of evaluated candidates. The main
	advantage of our proposed algorithm over BR is that CT\_EXT does not evaluate the typical condition,
	which leads to a runtime reduction in most cases.	
%
\section{Basic Concepts}
%
	Let $DS$ be a dataset with $k$ objects described by $n$ condition attributes (features) and groped in
	in $r$ classes by a decision attribute. Every attribute in the set of condition attributes 
	$R=\lbrace x_1,...,x_n \rbrace$, hereinafter referred to as attributes, may be of any type and must
	have a defined comparison criterion (usually the simple equality). Let $DM$ be the binary discernibility
	matrix obtained from comparing every pair of objects in $DS$ belonging to different classes. Every 
	comparison of a pair of objects adds a row to $DM$ with (0=indiscernible,1=discernible) in the corresponding 
	attribute position (column). $DM$ has $m$ rows and $n$ columns. Comparisons generating a row with only 
	0's, hereinafter referred to as empty row, imply that two objects from different classes are indistinguishable 
	by their attributes values. These rows are called inconsistencies of $DS$ and are not included in $DM$.
	
	\begin{definition}\label{def:testor}
		Let $T \subseteq R$ be a subset of attributes form DS. We say that T is a testor if in the sub-matrix
		of DM formed by the columns corresponding to attributes in T, there is not any empty row.
	\end{definition}
	
	An empty row in this sub-matrix of $DM$ imply that a reduced dataset obtained from $DS$, using only those 
	attributes in $T$, has more inconsistencies than $DS$. 	
	
	Usually the number of rows in $DM$ ($m$) is large. Lazo-Cort\'es et al.~(\cite{Lazo2001}) proposed a reduction
	of $DM$ without loosing relevant information. They proved that this reduced matrix, called basic matrix, and
	$DM$; has the same set of testors. 
	
	\begin{definition} \label{def:BM}
		Let $S_{j} \subseteq R$ be a subset of attributes associated to the row j of DM, such that $x_i \in S_{j}$
		iff the row j has a 1 in the column i. And let $SS=\lbrace S_1, S_2,...,S_m  \rbrace$ be the set of
		possible subsets $S_{j}$ in DM. The Basic Matrix BM of DS is formed by all the rows in DM (without
		repetitions) for which its associated $S_{j}$ has not any proper subset in SS.
	\end{definition}
	
	\begin{table}[!htb]
    \begin{minipage}{.5\linewidth}
      \caption{Discernibility Matrix}
      \centering
        \begin{tabular}{ccccc}\label{tab:DM}
            $x_0$ & $x_1$ & $x_2$ & $x_3$ & $x_4$\\
        		\hline
        		1&1&0&0&0\\
        		0&1&0&0&1\\
        		1&0&0&1&0\\
        		0&0&1&0&1\\
        		0&1&1&0&1\\
        \end{tabular}
    \end{minipage}%
    \begin{minipage}{.5\linewidth}
      \centering
        \caption{Basic Matrix}
        \begin{tabular}{ccccc}\label{tab:BM}
            $x_0$ & $x_1$ & $x_2$ & $x_3$ & $x_4$\\
        		\hline
        		1&1&0&0&0\\
        		0&1&0&0&1\\
        		1&0&0&1&0\\
        		0&0&1&0&1\\
        		\\
        \end{tabular}
    	\end{minipage} 
	\end{table}
	
	We can then, substitute $DM$ by $BM$ in definition~\ref{def:testor} without 
	any loose of generality. An example $DM$ is shown in table~\ref{tab:DM} and
	its corresponding $BM$ can be seen in table~\ref{tab:BM}.
	
	\begin{definition}\label{def:TT}
		A subset of attributes $T \subseteq R$ is a typical testor iff T is a testor and
		$\forall x_i \in T, T \setminus x_i$ is not a testor. 
	\end{definition}	
%	
\subsection{Concepts for CT\_EXT}
%
	%An important concept in the CT\_EXT algorithm is the contribution.
		
	\begin{definition}\label{def:contrib}
		Given $T \subseteq R$ and $x_i \in R$ such that $x_i \notin T$. We say that $x_i$ contributes to T iff 
		the sub-matrix of BM formed with only those attributes in T has more empty rows than that formed with attributes
		in $T \cup \lbrace x_i \rbrace$.
	\end{definition}	
	
	The core of the CT\_EXT algorithm is supported by propositions~\ref{prop1} and~\ref{prop2}; which are stated in
	(Sanchez-D\'iaz et al.~\cite{Sanchez2010}, Theorems 1 and 2) respectively.
	
	\begin{proposition}\label{prop1} 
		Given $T \subseteq R$ and  $x_i \in R$ such that $x_i \notin T$. If $x_i$ does not contribute to T, then 
		$T\cup\{x_i\}$ cannot be a subset of any typical testor.
	\end{proposition}

	\begin{proposition}\label{prop2} 
		Given $T \subseteq R$ and $Z \subseteq R$ such that $Z \cap T = \emptyset$. If T is a testor, then $T \cup
		Z$ is a 	testor too, but it is not a typical testor.
	\end{proposition}
	
	For a fast implementation of the CT\_EXT algorithm, columns in $BM$ are coded as binary words with as many 
	bits as 	rows has $BM$. The cumulative mask for an attribute $x_i$ is denoted as $cm_{x_i}$ and is the binary
	word representing the ith column in $BM$. The cumulative mask for a subset of attributes $T=\lbrace
	x_{i1},x_{i2},...,x_{ik} \rbrace$ is defined	as $cm_T = cm_{x_{i1}} \vee cm_{x_{i2}} \vee ... \vee cm_{x_{ik}}$
	where $\vee$ stands for the binary OR operator. It is not hard to see that the number of 0's in $cm_T$ equals
	the number of empty rows in $BM$. According to definition~\ref{def:contrib}, $x_i$ contributes to $T$ iff
	$cm_{T\cup x_i}$ has more 1's than $cm_T$. The incremental nature of fast-CT\_EXT is given by the associative
	property of the OR operation, such that $cm_{T\cup x_i}=cm_T\vee cm_{x_i}$. Notice from this last formulation,
	that $x_i$ contributes to $T$ iff $cm_{T\cup x_i}\neq cm_T$ since $cm_{T\cup x_i}$ cannot have less 1's than
	$cm_T$. Finally, substituting $DM$ by $BM$ in definition~\ref{def:testor}, it is easy to see that $T \subseteq
	R$ is a testor iff $cm_T=(1,...,1)$ (has a 1 in every bit).


\section{The Recursive Implementation of CT\_EXT}
%
	The first step in the CT\_EXT algorithm is the sorting of the BM in order to reduce the search space. One of 
	the rows of $BM$ having the minimum number of 1's is moved up to the first row. Then, columns are rearranged 
	so that those columns having a 1 in the first row, stay in the leftmost side of the $BM$.  A pseudocode for
	the recursive implementation of CT\_EXT is shown in algorithm~\ref{alg1}. 
%	A problem with this recursive implementation is that we must store cumulative masks while the recursive
%	functions walks down to the base case. A worst case requires $\frac{n(n-1)m}{2}$ bits which, for a $BM$
%	with $n=1000$ attributes and $m=100000$ rows, rises to 5.8GB. 

	The algorithm execution starts with the first element (column) in the sorted $BM$ and repeats until an 
	attribute with a 0 in the first row is reached (line~\ref{line:firstrow}). Notice that for the rest of the
	candidates the first row will always be an empty row. If this single attribute is a testor (line~\ref{line:singleT})
	then it is saved in the testors set $TS$; else, the recursive evaluator is called with the current attribute
	as the base set $B$ and the remaining attributes to its right as the tail set $C$ (line~\ref{line:eval}).
	In the recursive function, every attribute in the tail is tested for contribution to the base set 
	(line~\ref{line:contrib}). The attributes are removed from the tail set for furthers evaluations, given
	that they are testors (line~\ref{line:remT}) or do not contribute to $B$ (line~\ref{line:remNoContrib}).
	This a priori evaluation and rejection of non-contributing and testor-forming attributes constitutes the key
	gain of our proposed implementation, and is supported by proposition~\ref{prop3}.
	Finally, the remaining attributes in the tail set, are concatenated with $B$ and used as base set for
	subsequent recursive evaluations with the remaining $C$ as tail (line~\ref{line:recursive}).

	\begin{proposition}\label{prop3} 
		Given $T \subseteq R$, $Z \subseteq R$ and  $x_i \in R$ such that $x_i \notin Z$ and $T \subseteq Z$. If 
		$x_i$ does not contribute to T or form a testor with T, then 	$Z\cup\{x_i\}$ cannot be a subset of any 
		typical testor.
	\end{proposition}	
	
	\begin{proof}
		As $T \subseteq Z$, $cm_T$ cannot have more 1's than $cm_Z$ by the definition of the cumulative mask.
		If $x_i$ does not contribute to $T$ then, $cm_{T \cup \lbrace x_i \rbrace}$ does not have more 1's than
		$cm_T$; and, by transitivity, $cm_{Z \cup \lbrace x_i \rbrace}$ cannot have more 1's than $cm_Z$. 
		Notice that  $\lbrace T \cup \lbrace x_i \rbrace\rbrace \subseteq \lbrace Z \cup \lbrace x_i \rbrace\rbrace$.
		If $T \cup \lbrace x_i \rbrace$ is a testor, $\lbrace Z \cup \lbrace x_i \rbrace\rbrace$ by 
		definition~\ref{def:TT} 	cannot be a typical testor, given that it is a superset of a testor.
	\qed
	\end{proof}
	
	\renewcommand{\algorithmicrequire}{\textbf{Input:}}
	\renewcommand{\algorithmicensure}{\textbf{Output:}}
	%\begin{multicols}{2}
	\begin{algorithm}
	\caption{Recursively calculate testors in $DM$}
	\label{alg1}
	\begin{multicols}{2}
	\begin{algorithmic}[1]
	  \REQUIRE Sorted $BM$
	  \ENSURE $TS$ - the complete set of testors
	  \STATE $i \Leftarrow 0$, $TS \Leftarrow \emptyset$
	  \WHILE {$BM(0,i) \neq 0$}\label{line:firstrow}
	  	\IF {$cm_{x_i} = (1,...,1)$}\label{line:singleT}
	  		\STATE $TS \Leftarrow TS\cup \lbrace x_i \rbrace$
	  	\ELSE
	  		\STATE eval($\lbrace x_i \rbrace$, $cm_{x_i}$, $\lbrace x_{i+1},..., x_m\rbrace$)\label{line:eval}
	  	\ENDIF
	  	\STATE $i \Leftarrow i+1$
	  \ENDWHILE
	  \item[] \item[] \item[] \item[] \item[]
	  \STATE eval($B$,$cm_B$,$C$)
	  \FORALL{$x \in C$} 
	  	\STATE $cm_{B\cup x}=cm_B \vee cm_x$
	  	\IF {$cm_{B\cup x}\neq cm_B$}\label{line:contrib}
	  		\IF {$cm_{B\cup \lbrace x\rbrace}=(1,...,1)$}
	  			\STATE $C \Leftarrow C\setminus x$\label{line:remT} 
	  			\STATE $TS \Leftarrow TS\cup\lbrace B\cup \lbrace x\rbrace \rbrace$
	  		\ENDIF
	  	\ELSE
	  		\STATE $C \Leftarrow C\setminus x$\label{line:remNoContrib} 
	  	\ENDIF
	  \ENDFOR
	  \FORALL{$x \in C$} 
	  	\STATE $C \Leftarrow C\setminus x$
	  	\STATE eval($B\cup \lbrace x\rbrace$,$cm_{B\cup \lbrace x\rbrace}$,$C$)\label{line:recursive} 
	  \ENDFOR
	\end{algorithmic}
	\end{multicols}
	\end{algorithm}
	
	
	The evaluated candidates by the CT\_EXT algorithm and our proposed implementation, on the basic matrix from
	table~\ref{tab:BM} are shown in table~\ref{tab:run}. Notice that while CT\_EXT evaluates 17 candidates,
	in our recursive implementation, only 13 candidates are evaluated.
	
	\begin{table}[!htb]\label{tab:run}\scriptsize
		\setlength{\tabcolsep}{.05em}
		\caption{Example execution}
      	\centering
    		\begin{tabular}{|lllll|lllll|}
    		\hline
    		\multicolumn{5}{|c|}{CT\_EXT} & \multicolumn{5}{c|}{Recursive CT\_EXT}\\
    		\hline
    		\{$ x_0$\} 	 		& \{$x_0,x_1,x_4$\}	& \{$x_0,x_3$\}		& \{$x_1,x_2,x_3$\}	& \{$x_1,x_4$\}	
    		& \{$x_0$\}			& \{$x_0,x_4$\} 		& \{$x_1,x_3$\} 		& \{$x_1,x_3,x_4$\}	&\\
    		\{$x_0,x_1$\} 		& \{$x_0,x_2$\}  	& \{$x_0,x_4$\}		& \{$x_1,x_2,x_4$\}	& 
    		& \{$x_0,x_1$\}		& \{$x_0,x_1,x_2$\} 	& \{$x_1,x_4$\}		&&\\
    		\{$x_0,x_1,x_2$\}	& \{$x_0,x_2,x_3$\}	& \{$x_1$\}			& \{$x_1,x_3$\}	  	&			
    		& \{$x_0,x_2$\} 		& \{$x_1$\}		 	& \{$x_1,x_2,x_3$\}	&&\\
    		\{$x_0,x_1,x_3$\} 	& \{$x_0,x_2,x_4$\}	& \{$x_1,x_2$\} 		& \{$x_1,x_3,x_4$\}	&			
    		& \{$x_0,x_3$\} 		& \{$x_1,x_2$\}	 	& \{$x_1,x_2,x_4$\}	&&\\
    		\hline
		\end{tabular}
	\end{table}
%
\section{Experiments}
%
%
\section{Conclusions}
%
% ---- Bibliography ----
%
\begin{thebibliography}{}
%


\bibitem[1997]{Ayaquica1997}
	Ayaquica, I. O.:
	Un nuevo algoritmo de escala exterior para el c\'alculo de testores t\'ipicos.
	Memorias del II Taller Iberoamericano de Reconocimiento de Patrones , La
	Habana. 141--148 (1997)

\bibitem[2001]{Lazo2001}	
	Lazo-Cort\'es, M., Ruiz-shulcloper, J., Alba-Cabrera, E.:
	An Overview of the Evolution of the Concept of Testor. 
	Pattern Recognition. 34, 753--762 (2001)

\bibitem[2009]{Lias2009}
	Lias-Rodr\'iguez, A., Pons-Porrata, A.:
	BR: A new method for computing all typical testors. 
	Lecture Notes in Computer Science (including Subseries Lecture Notes 
	in Artificial Intelligence and Lecture Notes in Bioinformatics).
	5856 LNCS, 433--440 (2009)

\bibitem[2013]{Lias2013}	
	Lias-Rodr\'iguez, A., Sanchez-D\'iaz, G.:
 	An Algorithm for Computing Typical Testors Based on Elimination of Gaps and Reduction of Columns.
 	International Journal of Pattern Recognition and Artificial Intelligence. 27(08), 1350022 (2013)

\bibitem[2001]{Martinez2001}
	Mart\'inez-Trinidad, J.F., Guzm\'an-Arenas, A.: 
	The Logical Combinatorial Approach to Pattern Recognition an Overview through Selected Works. 
	Pattern Recognition. 34, 741--751 (2001)

\bibitem[2007]{Sanchez2007}
	Sanchez-D\'iaz, G., Lazo-Cort\'es, M.:
	CT-EXT: an algorithm for computing typical testor set. 
	In Progress in Pattern Recognition, Image Analysis and Applications. Springer. 506--514 (2007)

\bibitem[2010]{Sanchez2010}
	Sanchez-D\'iaz, G., Piza-Davila, I., Lazo-Cort\'es, M., Mora-Gonz\'alez, M., Salinas-Luna, J.:
	A fast implementation of the CT-EXT algorithm for the testor property identification. 
	Lecture Notes in Computer Science (including Subseries Lecture Notes in Artificial Intelligence and 
	Lecture	Notes in Bioinformatics), 6438 LNAI(PART 2), 92--103 (2010)

\bibitem[2003]{Santiesteban2003}	
	Santiesteban, Y., Pons-Porrata, A.:
	LEX: a new algorithm for the calculus of typical testors. 
	Mathematics Sciences Journal, 21(1), 85--95 (2003)

\bibitem[1995a]{Shulcloper1995}
	Ruiz-Shulcloper, J., Alba-Cabrera, E., Lazo-Cort\'es, M.:
	Introducci\'{o}n al Reconocimiento de Patrones (Enfoque L\'{o}gico--Combinatorio). 
	Serie Verde No. 51. CINVESTAV-IPN, México, (1995)

\bibitem[1995b]{Shulcloper1995b}
	Ruiz-Shulcloper, J., Alba-Cabrera, E., Lazo-Cort\'es, M.:
	Introducci\'{o}n a la teor\'ia de Testotes T\'ipicos. 
	Serie Verde No. 50. CINVESTAV-IPN, México, (1995)
	
\bibitem[1985]{Shulcloper1985}	
	Ruiz-Shulcloper, J., Aguila, L., Bravo, A.:
	BT and TB algorithms for computing all irreducible testors. 
	Revista Ciencias Matem\'{a}ticas, 2, 11--18 (1985)

\bibitem[2008]{Shulcloper2008}
	Ruiz-Shulcloper, J.:
	Pattern recognition with mixed and incomplete data. 
	Pattern Recognition and Image Analysis. 18(4), 563--576 (2008)

\bibitem[1992]{Skowron1992}
	Skowron, A., Rauszer, C.:
	The discernibility matrices and functions in information systems. 
	Handbook of Applications and Advances of the Rough Sets Theory. 331--362  (1992)
	
	
\end{thebibliography}

%\clearpage
%\addtocmark[2]{Author Index} % additional numbered TOC entry
%\renewcommand{\indexname}{Author Index}
%\printindex
%\clearpage
%\addtocmark[2]{Subject Index} % additional numbered TOC entry
%\markboth{Subject Index}{Subject Index}
%\renewcommand{\indexname}{Subject Index}
%\input{subjidx.ind}
\end{document}
