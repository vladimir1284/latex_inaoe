% This is LLNCS.DEM the demonstration file of
% the LaTeX macro package from Springer-Verlag
% for Lecture Notes in Computer Science,
% version 2.4 for LaTeX2e as of 16. April 2010
%
\documentclass[citeauthoryear]{llncs}
%
%\usepackage{makeidx}  % allows for indexgeneration
%
\usepackage{algorithm,algorithmic}
%\usepackage{hyperref}
%\hypersetup{colorlinks=true}	    % colores en vez de cajas en los enlaces
\usepackage{multicol}
\usepackage{graphicx}           	% para manejar imagenes
\usepackage[utf8]{inputenc}

\begin{document}
\mainmatter              % start of the contributions
%
\title{Fast--BR and fast--CT\_EXT performance regarding basic matrix properties: an empirical study.}
%
\titlerunning{Recursive CT_EXT}  % abbreviated title (for running head)
%                                     also used for the TOC unless
%                                     \toctitle is used
			 
\author{Vlad\'{i}mir Rodr\'{i}guez\inst{1,2} \and Jos\'{e}~F. Mart\'{i}nez\inst{1}
		 \and Jes\'{u}s~A. Carrasco\inst{1} \and Manuel~S.~Lazo\inst{1}}
%
\authorrunning{Vlad\'{i}mir Rodr\'{i}guez et al.} % abbreviated author list (for running head)
%
%%%% list of authors for the TOC (use if author list has to be modified)
%\tocauthor{Ivar Ekeland, Roger Temam, Jeffrey Dean, David Grove,
%Craig Chambers, Kim B. Bruce, and Elisa Bertino}
%
\institute{Instituto Nacional de Astrof\'{i}sica, \'{O}ptica y Electr\'{o}nica,\\
		   Luis Enrique Erro \# 1, Tonantzintla, Puebla, M\'{e}xico,\\
		   Coordinaci\'{o}n de Ciencias Computacionales,\\
\email{vladimir.rodriguez@ccc.inaoep.mx}
\and Universidad de Camag\"{u}ey,\\
	 Circunvalaci\'{o}n Nte. km 5$\frac{1}{2}$, Camag\"{u}ey, Cuba}


\maketitle              % typeset the title of the contribution

\begin{abstract}
	The Testors Theory is an important approach to feature selection in supervised classification. Typical testors are irreducible subsets of features preserving the discernibility between objects in the original dataset. A new dataset using only those features in a typical testor is a reduced representation of the original one, which improves the efficiency of machine learning tools without degrading their performance. Finding the complete set of typical testors for a dataset requires a high computational effort. Resent research has unveiled the relation of typical testors finding algorithms' performance with some properties of the basic matrix. In this paper we make an empirical study involving two of the most recent and faster algorithms in the state of the art: fast--BR and fast--CT\_EXT. Our study is carried out on synthetically generated basic matrices, and results in a simple rule for selecting a priori the fastest algorithm for a given problem. Finally our empirical rule is evaluated on standard datasets.

\keywords{Testors Theory, Algorithms, Basic matrix properties}
\end{abstract}
%
\section{Introduction}
%
	Feature selection is an important task in supervised classification. An information system is a dataset (table) containing objects (rows) which are	characterized by the value of some features (columns). A reduced representation of an information system, with the same discernibility between objects than that of the original dataset, reduces the computational	cost of the classification process. In the Logical Combinatorial Pattern Recognition (Ruiz-Shulcloper et al.~\cite{Shulcloper1995}), Testors Theory emerges as a solution to feature selection (Ruiz-Shulcloper~\cite{Shulcloper2008};	Mart\'inez-Trinidad and	Guzm\'an-Arenas~\cite{Martinez2001}). A testor is subset of features which allows us to discern between objects in different classes by using only values of its features. A Typical Testor (TT) is defined as a testor which is minimal with respect to inclusion. The main limitation to the application of the Testors Theory is that finding all the TT has exponential complexity regarding the number of attributes in the dataset (González-Guevara et al.~\cite{Gonzalez15}).
	
	One of the first algorithm designed to overcome the exponential complexity (regarding the number of features) of the problem of finding all the TT, was proposed by Ruiz-Shulcloper et al.~(\cite{Shulcloper1985}). This algorithm, called BT, codified a subset of features as a binary word with as many bits as features in the dataset. A 0 represents the absence of the corresponding feature in the current	subset while a 1 represents its inclusion. This way, candidates subsets are evaluated in the natural order of binary numbers. The pruning process in the	search space is based on the minimal condition of TT and a convenient sorting of the basic matrix (see definition~\ref{def:BM}) associated to the dataset. Finally, testors found by BT algorithm must by compared with each other in order to remove	any superset (not a TT by definition). In (Ruiz-Shulcloper et al.~\cite{Shulcloper1995b}) a new algorithm (REC) is presented.	The main drawback of REC was that it operated directly over the dataset (instead of the	basic matrix), handling a huge amount of superfluous information. Ayaquica~(\cite{Ayaquica1997})	presented the algorithm CER directed to solve this problem by using a different traversing	order. 
	
	Then, Santiesteban and Pons-Porrata~(\cite{Santiesteban2003}) proposed a new algorithm called LEX. The main ideas behind LEX are a new traversing order of candidates (which resembles the	lexicographical order in which string characters are compared) and the concept of gap. In LEX the typical condition is verified first and only for those potentially TT, the testor condition is checked. This way, the out-coming testors from this algorithm are always typical. The concept of gap allows us; once obtained a TT (or a not testor) candidate, including the last feature in the dataset, avoid the evaluation of any subset of this candidate.
	
	Sanchez-D\'iaz and Lazo-Cort\'es~(\cite{Sanchez2007}) proposed the CT\_EXT algorithm for computing all TT. Following a traversing order similar to that in LEX, this algorithm search for testors without verifying the typical condition. In this way, a larger number of candidates are evaluated, in comparison to LEX; but the cost of each evaluation is lower. Results from experiments	show that CT\_EXT is faster than the previous existing algorithm for most datasets. Then, Lias-Rodr\'iguez and Pons-Porrata~(\cite{Lias2009}) presented the BR algorithm, a Recursive algorithm based on Binary operations. BR is very similar to LEX in its bones but its recursive nature encloses a reduction in the number of evaluated candidates. Given a candidate subset, the remaining features are tested a priori and those being rejected are excluded from subsequent evaluations. Sanchez-D\'iaz et al.~(\cite{Sanchez2010}) presented a cumulative procedure for the CT\_EXT algorithm. This fast-CT\_EXT implementation reduces drastically the runtime for most datasets at no extra cost. In (Lias-Rodr\'iguez and Sanchez-D\'iaz~\cite{Lias2013}) the gap elimination and column reduction are added to BR. The main drawback of fast-BR and BR is, as in LEX, the high cost of evaluating the typical condition for every contributing candidate. 
	
	We have described above the evolution of the so called \emph{external scale} algorithms for typical testor computation. In addition, a different line of algorithms such as CT (Bravo-Martínez~\cite{Bravo83}), CC (Águila and Ruíz-Shulcloper~\cite{Aguila84}) and YYC (Alba-Cabrera et al.~\cite{Alba14}) have being developed. These are called \emph{internal scale} algorithms, and they analyse the basic matrix to find out some conditions to guarantee that a subset of attributes is a typical testor. Internal scale algorithms usually evaluate less candidates than external scale algorithms but each candidate evaluation has a higher computational cost. Therefore, the search for fast algorithms for computing typical testors has been biased to external scale algorithms (Alba-Cabrera et al.~\cite{Alba14}).
	
	Recently, a thorough study presented by Alba-Cabrera et al.~(\cite{Alba13}) concluded that no single typical testor finding algorithm have the best performance for any given problem. Other studies (Lias-Rodr\'iguez and Sanchez-D\'iaz~\cite{Lias2013}, Rodríguez-Diez et al.~\cite{Rodriguez15}), categorize basic matrices by the density of 1's they have; i.e. the number of ones divided by the total number of cells of the matrix. Furthermore, González-Guevara et al.~(\cite{Gonzalez15}) referred a series of elements that influence the algorithms' performance, such as the number of rows, the density of 1's and the number of typical testor of the basic matrix. With this precedents, we present in this paper a empirical study to identify a relationship between some of these properties of the basic matrix and the relative performance of algorithms. For this purpose, we selected fast--CT\_EXT (Sanchez-D\'iaz et al.~\cite{Sanchez2010}) and fast--BR (Lias-Rodr\'iguez and Sanchez-D\'iaz~\cite{Lias2013}) which are two of the most recent and faster algorithms in the state of the art. From this empirical study we obtained a simple rule to determine a priori the fastest algorithm for a determined problem. Finally we evaluated our rule on standard datasets from UCI (Bache and Lichman~\cite{Bache13}). 
	
%
\section{Basic Concepts}
%
	In this section we introduce the main propositions and definitions supporting the pruning strategies of fast--CT\_EXT and fast--BR. Here, we aim to provide the key elements to understand the differences between these two algorithms. Also, we introduced the necessary concepts to make this section self-contained.	
	
	Let $DS$ be a dataset with $k$ objects described by $n$ condition attributes (features) and groped in $r$ classes by a decision attribute. Every attribute in the set of condition attributes $R=\lbrace x_1,...,x_n \rbrace$, hereinafter referred to as attributes, may be of any type and must have a defined comparison criterion (usually the simple equality). Let $DM$ be the binary discernibility matrix obtained from comparing every pair of objects in $DS$ belonging to different classes. Every comparison of a pair of objects adds a row to $DM$ with (0=indiscernible,1=discernible) in the corresponding attribute position (column). $DM$ has $m$ rows and $n$ columns. Comparisons generating a row with only 0's, hereinafter referred to as empty row, imply that two objects from different classes are indistinguishable by their attributes values. These rows are called inconsistencies of $DS$ and are not included in $DM$.
	
	\begin{definition}\label{def:testor}
		Let $T \subseteq R$ be a subset of attributes form $DS$. We say that T is a testor if in the sub-matrix
		of DM formed by the columns corresponding to attributes in T, there is not any empty row.
	\end{definition}
	
	An empty row in this sub-matrix of $DM$ imply that a reduced dataset obtained from $DS$, using only those attributes in $T$, has more inconsistencies than $DS$. 	
	
	Usually the number of rows in $DM$ ($m$) is large. Lazo-Cort\'es et al.~(\cite{Lazo2001}) proposed a reduction of $DM$ without loosing relevant information. They proved that this reduced matrix, called \textit{basic matrix}, and $DM$; has the same set of testors. 
	
	\begin{definition} \label{def:BM}
		Let $S_{j} \subseteq R$ be a subset of attributes associated to the row j of DM, such that $x_i \in S_{j}$ iff the row j has a 1 in the column i. And let $SS=\lbrace S_1, S_2,...,S_m  \rbrace$ be the set of possible subsets $S_{j}$ in DM. The Basic Matrix BM of DS is formed by all the rows in DM (without repetitions) for which its associated $S_{j}$ has not any proper subset in SS.
	\end{definition}
	
	
	We can then, substitute $DM$ by $BM$ in definition~\ref{def:testor} without any loose of generality. 
	
	\begin{definition}\label{def:TT}
		A subset of attributes $T \subseteq R$ is a typical testor iff T is a testor and $\forall x_i \in T, T \setminus x_i$ is not a testor. 
	\end{definition}
		
%	
\subsection{Concepts for fast--CT\_EXT}
%
		
	\begin{definition}\label{def:contrib}
		Given $T \subseteq R$ and $x_i \in R$ such that $x_i \notin T$. We say that $x_i$ contributes to T iff the sub-matrix of BM formed with only those attributes in T has more empty rows than that formed with attributes in $T \cup \lbrace x_i \rbrace$.
	\end{definition}	
	
	The core of the CT\_EXT algorithm is supported by propositions~\ref{prop:contrib} and~\ref{prop:superset}; which are stated in	(Sanchez-D\'iaz et al.~\cite{Sanchez2010}, Theorems 1 and 2) respectively.
	
	\begin{proposition}\label{prop:contrib} 
		Given $T \subseteq R$ and  $x_i \in R$ such that $x_i \notin T$. If $x_i$ does not contribute to T, then 		$T\cup\{x_i\}$ cannot be a subset of any typical testor.
	\end{proposition}

	\begin{proposition}\label{prop:superset} 
		Given $T \subseteq R$ and $Z \subseteq R$ such that $Z \cap T = \emptyset$. If T is a testor, then $T \cup Z$ is a 	testor too, but it is not a typical testor.
	\end{proposition}

%
\subsection{Concepts for fast--BR}
%
	In addition to the propositions exposed above, fast--BR is supported by the following propositions; which are stated and proved in (Lias-Rodr\'iguez and Sanchez-D\'iaz~\cite{Lias2013}).

%	\begin{proposition}\label{prop:recursive} 
%		Given $T \subseteq R$, $Z \subseteq R$ and  $x_i \in R$ such that $x_i \notin Z$ and $T \subseteq Z$. If $x_i$ does not contribute to T or form a testor with T, then $Z\cup\{x_i\}$ cannot be a subset of any typical testor.
%	\end{proposition}	
%	
%	Proposition~\ref{prop:recursive} constitutes the basis for the recursive implementation of the algorithm. Fast--BR introduces a kind of look ahead variant to the lexicographical order to take advantage of this proposition. This procedure avoids subsequent evaluations of a non contributing attribute.
	
	\begin{definition}\label{def:exclusion}
		Given $T \subseteq R$ we call compatibility mask of $T$, denoted as $cm_T$, to the binary word in which the $j^{\mathit{th}}$ bit is 1 if the $j^{\mathit{th}}$ row of $BM$ has a 1 in only one column of those columns corresponding to attributes in $T$, and it is 0 otherwise.
	\end{definition}
	
	\begin{proposition}\label{prop:exclude} 
		Given $T \subseteq R$ and $x_i \in R$ such that $x_i \notin T$.	We denote $c_{x_k}$ to the binary word in which the $j^{\mathit{th}}$ bit is 1 if the $j^{\mathit{th}}$ row of $BM$ has a 1 in the column corresponding to $x_k$. If $\exists x_k \in T$ such that $cm_{T \cup \lbrace x_i\rbrace} \wedge c_{x_k}=(0,...,0)$. Then, $T \cup \lbrace x_i\rbrace$ do not form a typical testor. And we will say that $x_i$ is exclusionary with $T$.
	\end{proposition}
	
	We will refer to Proposition~\ref{prop:exclude} as exclusion evaluation. The application of exclusion evaluation for typical testor property identification is given in the form of Proposition~\ref{prop:TT}.
	
	\begin{proposition}\label{prop:TT} 
			Given $T \subseteq R$ and $x_i \in R$ such that $x_i \notin T$. The subset $T \cup \lbrace x_i\rbrace$ is a typical testor iff it is a testor and $x_i$ is not exclusionary with $T$.
	\end{proposition}
		
%
\section{Comparative Study}
%
	We include in our comparative study the main exponents of the two families of external scale algorithms: Those evaluating first the testor condition and then verify the typical condition (exclusion evaluation) of testors, and those evaluating the exclusion first and then the testor condition over non exclusionary subsets. This can be seen in the candidate evaluation process of fast--CT\_EXT and fast--BR illustrated in Fig.~\ref{fig:candeval}. Since the number of testors is usually a small fraction of the total evaluated candidates, we can state that fast--BR makes more exclusion evaluations than fast--CT\_EXT most of the times.
	
		%TODO hacer más gruesas las líneas de la figura
	\begin{figure}[htb]
	    \centering
	    \begin{minipage}{.5\textwidth}
	        \centering
	        \includegraphics[height=5cm]{fastct_ext.eps}
	    \end{minipage}%
	    \begin{minipage}{0.5\textwidth}
	        \centering
	        \includegraphics[height=5cm]{fastBR.eps}	        
	    \end{minipage}
		\caption{Candidate evaluation flowchart for fast--CT\_EXT (left) and fast--BR (right).}
		\label{fig:candeval}
	\end{figure}
		
	For our experiments in the next section we will use the implementation of fast--BR provided by their authors. This algorithm takes advantage of two concepts: gap elimination and columns reduction (Lias-Rodr\'iguez and Sanchez-D\'iaz~\cite{Lias2013}). In order to provide leverage to the comparative study, we included these procedures in the fast--CT\_EXT implementation that we will use in our experiments\footnote{The source code as well as all the basic matrices and datasets used in our experiments can be downloaded from \url{http://ccc.inaoep.mx/$\sim$ariel/CTBRES}}. In this way we can unveil the effects of the main difference between these two algorithm in their relative performance.
	
%
\subsection{Comparison over synthetic basic matrices}
%
In order to explore the influence of the basic matrix density on the relative performance of fast--CT\_EXT and fast--BR; we conducted this first experiment over 500 randomly generated basic matrices with 2000 rows and 30 columns. The size of these matrices was selected in order to keep reasonable runtime for the three algorithms. Our 500 matrices were generated with densities of 1's uniformly distributed in the range (0.20--0.80) using a step of 0.04. 

In Fig.~\ref{fig:reducts} we show the runtime of fast--CT\_EXT and fast--BR as a function of the number of typical testors in the basic matrix. We can see a positive correlation between these variables, but this information cannot be used a priori because computing the number of typical testors has the same complexity that finding them.

	\begin{figure}[htb]
	    \centering
	    \begin{minipage}{.5\textwidth}
	        \centering
	        \includegraphics[height=4cm]{runtime_vs_reducts.eps}
	        \caption{Runtime vs. number of reducts.}	   
	        \label{fig:reducts}
	    \end{minipage}%
	    %{~~}
	    \begin{minipage}{0.5\textwidth}
	        \centering
	        \includegraphics[height=4cm]{2000x30.eps}
	        \caption{Runtime vs. density.}    
	        \label{fig:density}
	    \end{minipage}
	\end{figure}

From Fig.~\ref{fig:density} it can be seen that fast--CT\_EXT was faster for basic matrices with density under or equal 0.35, while fast--BR was faster for matrices with density above 0.35. In order to explain this behavior, we must look back to Fig.~\ref{fig:candeval}. The exclusion evaluation has the higher computational complexity in the algorithm ($\Theta (nm)$). Exclusionary attributes are found when there is at least one column in the basic matrix, considering only those attributes in the current candidate, that can be removed without increasing the number of zero rows. This condition is more frequent in matrices with a higher density, where 1's overlapping is most likely. For matrices with a relative high density (the boundary identified from our experiment is 0.35), the heavier candidate evaluation process of fast--BR pays off, because exclusionary attributes are avoided from subsequent evaluations. For matrices with a lower density, the simples approach of fast--CT\_EXT results in a faster execution. Take for instance the extreme case of the identity matrix, where there is no any exclusionary attribute, since they are all indispensable to form a typical testor. For this kind of basic matrices, fast--CT\_EXT needs to evaluate as many candidates as fast--BR but the former makes a single exclusion evaluation with the set of all attributes. On the other hand, fast--BR evaluates the exclusion for each candidate, which leads to a higher computational cost.

	\begin{table}[htb]
		\centering 
		\setlength{\tabcolsep}{12pt}
		\caption{Boundary density for different number of rows in the basic matrix.}
		\label{tab:boundary}
		\begin{tabular}{lccccc}
			\hline
			$BM$ size & 2000 & 1000 & 500 & 250 & 125 \\
			\hline
			Boundary  & 0.35 & 0.31 & 0.28 & 0.24 & $<$0.20 \\		
			\hline
		\end{tabular}
	\end{table}
	
	In order to evaluate the influence of the number of rows in the basic matrix over the relative performance of the two algorithms, we repeated this first experiment for matrices with different number of rows. Each time, 500 basic matrices with the same number of rows was generated with 30 columns and a density distribution similar to that of our first experiment. The number of rows and the obtained boundary density for each experiment, are summarized in Table~\ref{tab:boundary}. 
	
	From these last experiments, we can estimate by a log-linear least-squares fitting that the boundary density $d_\beta$ is related to the number of rows in the basic matrix $m$ by $d_\beta=0.041\log_2(m)-0.095$.
%
\subsection{Evaluation on standard datasets}
%
	In order to evaluate the rule obtained over synthetic matrices, we selected 10 datasets from UCI (Bache and Lichman~\cite{Bache13}). Table~\ref{tab:density} shows the runtime for fast--CText and fast--BR over our selected datasets. Notice that we chose 5 datasets with basic matrix densities lower or equal 0.35 and 5 datasets above. Although this experiment was carried out on a small heterogeneous sample, it provides evidence on the applicability of the rule for selecting the appropriated algorithm for a given problem.
	
	\begin{table}[htb]
		\centering \footnotesize
		\caption{Fast--CText and fast--BR runtimes for standard datasets. Sorted by basic matrix density.}
		\label{tab:density}
		\begin{tabular}{lcccccrr}
			\hline
			&&&& \multicolumn{2}{c}{Basic Matrix} &  \multicolumn{1}{c}{Fast--CText} & \multicolumn{1}{c}{Fast--BR} \\
			\cline{5-6}
			Dataset   		 & Attributes & Instances & Reducts  & Rows  & Density & runtime (s) & runtime (s) \\
			\hline
			Keyword-activity & 37         & 1530      & 3        & 26    & 0.04    & \textbf{0.42}    & 0.90            \\
			Connect-4        & 43         & 6756      & 35       & 406   & 0.05    & \textbf{44.23}   & 160.61          \\
			QSAR-biodeg      & 42         & 1055      & 256      & 40    & 0.12    & \textbf{0.19}    & 0.33            \\
			Credit-g         & 21         & 1000      & 846      & 223   & 0.35    & \textbf{0.06}    & 0.12            \\
			Flags            & 30         & 194       & 23543    & 390   & 0.35    & \textbf{0.74}    & 1.06            \\
			Student-por      & 32         & 649       & 851584   & 8158  & 0.41    & 1657.90          & \textbf{161.35} \\
			Sponge           & 46         & 76        & 10992    & 68    & 0.42    & 0.58             & \textbf{0.14}   \\
			Student-mat      & 32         & 395       & 679121   & 6904  & 0.43    & 929.46           & \textbf{81.82}  \\
			Lung-cancer      & 57         & 32        & 4183355  & 237   & 0.47    & 133.43           & \textbf{7.34}   \\
			Cylinder-bands   & 40         & 512       & 23534    & 1147  & 0.55    & 4.59             & \textbf{0.53}   \\
			\hline
		\end{tabular}
	\end{table}
%
%
\section{Conclusions}
%
% ---- Bibliography ----
%
\begin{thebibliography}{}
%


\bibitem[1984]{Aguila84}	
	Águila, L., Ruíz-Shulcloper, J.  
	Algoritmo CC para la elaboración de la información k-valente en problemas de Reconocimiento de Patrones. 
	Mathematics Sciences Journal (In Spanish), 
	5(3) (1984)
	
\bibitem[2013]{Alba13}
	Alba-Cabrera, E., Ibarra-Fiallo, J.,Godoy-Calderon, S.:
	A Theoretical and Practical Framework for Assessing the Computational Behavior of Typical Testor-Finding Algorithms.
	In CIARP 2013, Part I. LNCS,
	8258, 351--358, (2013)
	
\bibitem[2014]{Alba14}
	Alba-Cabrera, E., Ibarra-Fiallo, J., Godoy-Calderon, S., Cervantes-Alonso, F.:
	YYC: A Fast Performance Incremental Algorithm for Finding Typical Testors.
	Progress in Pattern Recognition, Image Analysis, Computer Vision, and Applications.
	416--423 (2014)
	
\bibitem[1997]{Ayaquica1997}
	Ayaquica, I. O.:
	Un nuevo algoritmo de escala exterior para el c\'alculo de testores t\'ipicos.
	Memorias del II Taller Iberoamericano de Reconocimiento de Patrones , La
	Habana. 141--148 (1997)
	
\bibitem[2013]{Bache13}	
	Bache, K., Lichman, M.:
	UCI machine learning repository.
	(2013)
	
\bibitem[1983]{Bravo83}	
	Bravo-Martínez, A.:
	Algorithm CT for Calculating the Typical Testors of k-valued Matrix. 
	Mathematics Sciences Journal (In Spanish), 
	4(2), 123--144 (1983)
	
\bibitem[2015]{Gonzalez15}		
	González-Guevara, V., Godoy-Calderón, S., Alba-Cabrera, E.,  Ibarra-Fiallo, J.:
	A Mixed Learning Strategy for Finding Typical Testors in Large Datasets. 
	In CIARP 2015. LNCS,
	5197, 716--723 (2008)
		
\bibitem[2001]{Lazo2001}	
	Lazo-Cort\'es, M., Ruíz-Shulcloper, J., Alba-Cabrera, E.:
	An Overview of the Evolution of the Concept of Testor. 
	Pattern Recognition. 34, 753--762 (2001)

\bibitem[2009]{Lias2009}
	Lias-Rodr\'iguez, A., Pons-Porrata, A.:
	BR: A new method for computing all typical testors. 
	Lecture Notes in Computer Science (including Subseries Lecture Notes 
	in Artificial Intelligence and Lecture Notes in Bioinformatics).
	5856 LNCS, 433--440 (2009)

\bibitem[2013]{Lias2013}	
	Lias-Rodr\'iguez, A., Sanchez-D\'iaz, G.:
 	An Algorithm for Computing Typical Testors Based on Elimination of Gaps and Reduction of Columns.
 	International Journal of Pattern Recognition and Artificial Intelligence. 27(08), 1350022 (2013)

\bibitem[2001]{Martinez2001}
	Mart\'inez-Trinidad, J.F., Guzm\'an-Arenas, A.: 
	The Logical Combinatorial Approach to Pattern Recognition an Overview through Selected Works. 
	Pattern Recognition. 34, 741--751 (2001)

\bibitem[2015]{Rodriguez15}	
	Rodríguez-Diez, V., Martínez-Trinidad, J. F., Carrasco-Ochoa, J. A., Lazo-Cortés, M., Feregrino-Uribe, C., Cumplido, R.:
	A fast hardware software platform for computing irreducible testors. 
	Expert Systems with Applications, 
	42(24), 9612–9619 (2015)

\bibitem[1995a]{Shulcloper1995}
	Ruíz-Shulcloper, J., Alba-Cabrera, E., Lazo-Cort\'es, M.:
	Introducci\'{o}n al Reconocimiento de Patrones (Enfoque L\'{o}gico--Combinatorio). 
	Serie Verde No. 51. CINVESTAV-IPN, México, (1995)

\bibitem[1995b]{Shulcloper1995b}
	Ruíz-Shulcloper, J., Alba-Cabrera, E., Lazo-Cort\'es, M.:
	Introducci\'{o}n a la teor\'ia de Testotes T\'ipicos. 
	Serie Verde No. 50. CINVESTAV-IPN, México, (1995)
	
\bibitem[1985]{Shulcloper1985}	
	Ruíz-Shulcloper, J., Aguila, L., Bravo, A.:
	BT and TB algorithms for computing all irreducible testors. 
	Revista Ciencias Matem\'{a}ticas, 2, 11--18 (1985)

\bibitem[2008]{Shulcloper2008}
	Ruíz-Shulcloper, J.:
	Pattern recognition with mixed and incomplete data. 
	Pattern Recognition and Image Analysis. 18(4), 563--576 (2008)
	
\bibitem[2007]{Sanchez2007}
	Sanchez-D\'iaz, G., Lazo-Cort\'es, M.:
	CT-EXT: an algorithm for computing typical testor set. 
	In Progress in Pattern Recognition, Image Analysis and Applications. Springer. 506--514 (2007)

\bibitem[2010]{Sanchez2010}
	Sanchez-D\'iaz, G., Piza-Davila, I., Lazo-Cort\'es, M., Mora-Gonz\'alez, M., Salinas-Luna, J.:
	A fast implementation of the CT-EXT algorithm for the testor property identification. 
	Lecture Notes in Computer Science (including Subseries Lecture Notes in Artificial Intelligence and 
	Lecture	Notes in Bioinformatics), 6438 LNAI(PART 2), 92--103 (2010)

\bibitem[2003]{Santiesteban2003}	
	Santiesteban, Y., Pons-Porrata, A.:
	LEX: a new algorithm for the calculus of typical testors. 
	Mathematics Sciences Journal, 21(1), 85--95 (2003)

%\bibitem[1992]{Skowron1992}
%	Skowron, A., Rauszer, C.:
%	The discernibility matrices and functions in information systems. 
%	Handbook of Applications and Advances of the Rough Sets Theory. 331--362  (1992)
	
	
\end{thebibliography}

%\clearpage
%\addtocmark[2]{Author Index} % additional numbered TOC entry
%\renewcommand{\indexname}{Author Index}
%\printindex
%\clearpage
%\addtocmark[2]{Subject Index} % additional numbered TOC entry
%\markboth{Subject Index}{Subject Index}
%\renewcommand{\indexname}{Subject Index}
%\input{subjidx.ind}
\end{document}
